%!TEX TS-program = xelatex
%!TEX encoding = UTF-8 Unicode

%%% Local Variables: 
%%% coding: utf-8
%%% mode: latex
%%% TeX-engine: xetex
%%% End: 

\documentclass[10pt,letterpaper,final]{moderncv}
\usepackage{fontspec}
\ifx\HCode\UnDef\else\hypersetup{tex4ht}\fi 
\moderncvtheme[darkred]{classic}

% DOCUMENT LAYOUT
\usepackage[scale=0.85]{geometry}
\setlength{\hintscolumnwidth}{2cm}
\AtBeginDocument{\recomputelengths}
% FONTS
\usepackage[utf8]{inputenc}
\defaultfontfeatures{Mapping=tex-text} % converts LaTeX specials (``quotes'' --- dashes etc.) to unicode

% Remove % to set to different fonts
%\setromanfont [Ligatures={Common},Numbers={OldStyle}]{Adobe Caslon Pro}
%\setmonofont[Scale=0.8]{Monaco} 

% ---- CUSTOM AMPERSAND
\newcommand{\amper}{{\fontspec[Scale=.95]{Adobe Caslon Pro}\selectfont\itshape\&}}

% ---- MARGIN YEARS
%\newcommand{\years}[1]{\marginpar{\scriptsize #1}}


% PDF SETUP
% ---- FILL IN HERE THE DOC TITLE AND AUTHOR


% Personal Information 
\firstname{Didier}
\familyname{Deshommes}
\address{600 N 7th St.}{McAllen, TX 78501}
\mobile{(919) 616-0344}
\email{dfdeshom@gmail.com}

\title{Didier Deshommes' Resume}

\nopagenumbers{}

\begin{document} 

\maketitle

\section{Executive Summary} 
  \cvline{Summary:}{Python engineer with experience in creating
    and implementing scalable solutions in RESTful
    API design, web-crawling, data processing/analytics pipelines, database schemas, and search.}
\section{Experience} 
  \cventry{Spring 2009 - February 2016}{Systems and architecture Engineer}{Parse.ly}{New York, NY}{}{
      \begin{itemize}
        \item Designed and created an API for our publishers to access their
          analytics and recommendations data. The API powers multiple
          live dashboards and widgets. It serves on average 500 requests per second
          and peaks up to 1000-1500 requests per second. The
          documentation for the API is here:
          \url{https://www.parsely.com/docs/api/overview/endpoint.html}
          . Technologies used: Tornado, Redis, MongoDB, ElasticSearch,
          Cassandra. 
        \item Designed and created the crawling system that processes millions of URLS per 
          month. Crawlers collect informative metadata from publishers' pages for 
          display on the Parse.ly's Dash analytics dashboard.  
        \item  Implemented several features in our real-time and batch
          analytics pipeline that processes hundreds of millions of events per
          day. The pipeline powers the Parse.ly's Dash analytics
          product. Technologies used: Pig, Spark, Storm,
          ElasticSearch, Cassandra.   
        \item Designed and implemented flexible auto-complete search for our
            dashboard analytics users. Users could complete either the beginning, middle or end of
            several words. Users were
            able to search millions of titles, authors and sections in
            a unified way  in less than 200ms. 
        \item Designed and developed named entity recognition system using Solr, 
          the Wikipedia index, Wikipedia traffic data, and NLTK. With
          this feature, users were able to see what topics were
          trending on the web based on the traffic they got from
          Wikipedia traffic.
        \item Automated  configuration and deployment for machines
          that were used to do API, crawling or search work using Chef
          and Vagrant.
        \item Wrote and launched the back-end prototype to the Parse.ly
          reader, an RSS feed reader that displays articles from the
          web based on a user's interests, using Django and Solr. 
      \end{itemize}
  }
  \cventry{Winter 2007 - Spring 2008}{Senior Python
    contractor}{Wordstream}{Boston, MA}{}{
    \begin{itemize}
      \item Designed and implemented a keyword-matching algorithm
        that accepts documents as its input and produces a list of
        suggested hyperlinks to be added by a user. The suggested
        keywords are based on the user's profile data. 
    \end{itemize}
  }
  \cventry{2007 - 2008}{Software Engineer}{Intelligent Information Systems}{Durham, NC}{}{
    \begin{itemize}
      % \item Helped design and implement a web application that allows
      %   some North Carolina counties to handle and process appeals
      %   related to taxes on property made by residents of these
      %   counties using C\#, ASP.NET, MS SQL and Javascript 
      \item Designed and implemented a website so clients could
        access and browse property inspection information 
      \item Implemented and improved a procedure to port and test
        programs written in VB6 to VB.NET. Converted over 100 programs
        written in VB6 code to VB.NET using a combination of the MS
        conversion tool and Python. 
    \end{itemize}
  }

\section{Education}
\cventry {Fall 2006 - Winter 2008}{MS Applied Mathematics, North
  Carolina State University} {Raleigh, NC}{}{}{}
\cventry{Fall 2001 - Winter 2006}{BS Computer Science, North Carolina State University}{Raleigh, NC}{}{}{}
\cventry{Fall 2001 - Winter 2006}{BS Applied Mathematics, North Carolina State University}{Raleigh, NC}{}{}{}

\section{Extra}
\cvlistitem{Presenter,PyData: ``Wikipedia Indexing and Analysis'' - see \url{https://vimeo.com/53091620}}
\cvlistitem{Contributor to open source projects - see
    \url{https://github.com/dfdeshom}}

\end{document}
