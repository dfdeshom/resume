%!TEX TS-program = xelatex
%!TEX encoding = UTF-8 Unicode

%%% Local Variables: 
%%% coding: utf-8
%%% mode: latex
%%% TeX-engine: xetex
%%% End: 

\documentclass[8pt,letterpaper,final]{moderncv}
\usepackage{fontspec}
\ifx\HCode\UnDef\else\hypersetup{tex4ht}\fi 
\moderncvtheme[darkred]{classic}

% DOCUMENT LAYOUT
\usepackage[scale=0.85]{geometry}
\setlength{\hintscolumnwidth}{2cm}
\AtBeginDocument{\recomputelengths}
% FONTS
\usepackage[utf8]{inputenc}
\defaultfontfeatures{Mapping=tex-text} % converts LaTeX specials (``quotes'' --- dashes etc.) to unicode

% Remove % to set to different fonts
%\setromanfont [Ligatures={Common},Numbers={OldStyle}]{Adobe Caslon Pro}
%\setmonofont[Scale=0.8]{Monaco} 

% ---- CUSTOM AMPERSAND
\newcommand{\amper}{{\fontspec[Scale=.95]{Adobe Caslon Pro}\selectfont\itshape\&}}
\newcommand{\fan}{\phantom{caption}}

% ---- MARGIN YEARS
%\newcommand{\years}[1]{\marginpar{\scriptsize #1}}


% PDF SETUP
% ---- FILL IN HERE THE DOC TITLE AND AUTHOR

% Personal Information 
\firstname{Didier}
\familyname{Deshommes}
\address{1413 Oriole Ave.}{McAllen, TX 78504}
\mobile{(919) 616-0344}
\email{dfdeshom@gmail.com}

\title{Didier Deshommes' Resume}

\nopagenumbers{}

\begin{document} 

\maketitle

\section{Executive Summary} 
  \cvline{Summary}{Python engineer who has designed and implemented large-scale production systems. Experienced with RESTful APIs, high-throughput web crawlers, and fast data processing pipelines for analytics and search workloads. Especially interested in systems where either data volume or velocity is high.
}
\cvline{Production Experience}{ Spark, Chef/Ansible, AWS/EMR,
  Elasticsearch, Redshift, Cassandra, Kafka, MongoDB,
  Redis, Solr, Pig, Storm, Django, Tornado, Scrapy framework}
\cvline{Languages}{Python, Java}
\section{Experience}
  \cventry{January 2017 - Present}{Data Engineer}{Restless Bandit}{San
  Francisco, CA}{}{
  \begin{itemize}
  \item Responsible for maintaining and improving our data pipeline
    using Spark and ElasticSearch. My focus has been on increasing the
    speed of out data-processing by optimizing critical part of our
    code (using Cython) and designing better ways to run our daily
    Spark jobs.
  \item Reduced cost of running daily EMR jobs by around 40\% in terms
    of normalized instance hours by analyzing job resource usage and selecting
    appropriate instances for each type of job.
  \item Created  internal  RESTful web APIs for use by our data
    pipelines using Python and Celery. Created various ansible
    playbooks for provisioning and deploying machines running these APIs.   
  \item Helped in the migration of Spark output data from CloudSearch to ElasticSearch
  \end{itemize}
}
  \cventry{May 2016 - January 2017}{Software Engineer}{Shareaholic}{Boston, MA }{}{
      \begin{itemize}
         \item Lead design and implementation of a system that built
          user profiles  from various attributes such as 
          geographic location, content  of pages visited, referrer
          data. These profiles could then be used for highly-specific
          targeting and data sales to various data partners. Wrote
          much of the tooling to make launching the system simple and
          automatic. The system has delivered thousands of additional
          revenue to data partners in its first 3 months existence.
        \item Lead engineering effort to revamp our 100-machine
          infrastructure managed by Chef. Re-wrote many of our custom
            recipes and roles, added more recipes to increase
            automation. Wrote a tool to make cloning or creating new
            machines from AWS simpler. 
       \end{itemize}
  }
  \cventry{ Spring 2011 - February 2016}{Backend Lead}{Parse.ly}{New
    York, NY (Fully Remote Team)}{}{
      \begin{itemize}
        \item  Lead implementation of major backend systems for an
          engineering team that grew from 3 to 20 over the course of 4
          years. During this period, the company also grew to hundreds
          of customers, millions in revenue and massive data
          volumes. An overview of Parse.ly's core product is available
          on its website.
        \item After seed financing round in 2011, company hired other engineers to
          focus on areas like dashboard design and time series data. Took full
          ownership over backend systems and APIs while company pivoted into the
          content analytics market, where it ended up finding
          success. 
          \item  Built our API, which allowed
            customers to access analytic and recommendations data. It
            was adopted by 40\% of customers and by 2015 was serving 2
            billion calls per month. It serves 500
            requests per second and peaks up to
            2,000 requests per second. It powers
            popular websites like The New Yorker and
            Arstechnica. Also wrote its public documentation:
            \url{https://www.parsely.com/docs/api/overview/endpoint.html} .
            An overview of how the recommendation engine works can be
            found here: \url{http://blog.parsely.com/post/3406/reco-engine/}
        % \item Designed and created an API for our publishers to access their
        %   analytics and recommendations data. The API powers multiple
        %   live dashboards and widgets. It serves on average 500 requests per second
        %   and peaks up to 1000-1500 requests per second. The
        %   documentation for the API is here:
        %   \url{https://www.parsely.com/docs/api/overview/endpoint.html}
        %   . Technologies used: Tornado, Redis, MongoDB, ElasticSearch,
        %   Cassandra.
          \item Created the crawling system that processes over 100
            million URLs per month. Crawlers collect informative
            metadata from publishers' pages for 
          display on the Parse.ly's Dash analytics dashboard.  
        \item  Implemented several features in our real-time and batch
          analytics pipeline that processes hundreds of millions of events per
          day. The pipeline powers the Parse.ly's Dash analytics
          product.    
        \item Designed and implemented flexible auto-complete search for our
            dashboard analytics users. Users could complete either the
            beginning, middle or end of 
            several words. Users were
            able to search millions of titles, authors and sections in
            a unified way  in less than 200ms. 
        \item Designed and developed named entity recognition system using Solr, 
          the Wikipedia index, Wikipedia traffic data, and NLTK. With
          this feature, users were able to see what topics were
          trending on the web based on the traffic they got from
          Wikipedia traffic.
        \item Automated  configuration and deployment for machines
          that were used to do API, crawling or search work using Chef
          and Vagrant.
          \item During period of high growth, performed several
            technology evaluations as company refreshed its technology
            stack for scale. Company raised series A financing in 2013
            and grew to a total headcount of over 40, while landing
            new customers that included the highest-traffic websites
            on the Internet, such as Reuters, The Huffington Post,
            Business Insider, and WIRED. Evaluated distributed systems
            like Elasticsearch and Cassandra, which were eventually
            adopted to replace MongoDB and Solr. 
      \end{itemize}
  }
  \cventry{Summer 2009 - Spring 2011}{Software Engineer \& Employee
    \#1}{Parse.ly}{New York, NY (Fully Remote Team)}{}{
    \begin{itemize}
      \item First engineering hire at an early-stage startup that was
        part of Dreamit Ventures 2009, a YCombinator-like startup
        accelerator program in Philadelphia, PA. 
      \item Wrote and launched the backend to the Parse.ly Reader,
        a news article reader (similar concept to Flipboard, but
        before Flipboard's launch) that displayed articles from the
        web based on a user interests.  Worked
        directly with CTO. A review of the Parse.ly Reader is here:
        \url{http://blog.louisgray.com/2009/08/parsely-spices-up-news-based-on-your.html} 
      \item Rewrote backend to support a content recommendation API
          for publishers, as company moved into the enterprise API
          space serving content publishers. These systems helped the
          company land its first customers. Worked as sole full-time
          engineer through Parse.ly's bootstrapping period. 
        \item  Company raised a seed financing round in 2011 and
          changed its focus, while re-using technology developed
          during this founding period.  
    \end{itemize}
  }
  \cventry{Winter 2007 - Spring 2008}{Python Consultant}{Wordstream}{Boston, MA}{}{
    \begin{itemize}
      \item Designed and implemented a keyword-matching algorithm
        that accepts documents as its input and produces a list of
        suggested hyperlinks to be added by a user. The suggested
        keywords are based on the user's profile data. 
    \end{itemize}
  }
  \cventry{2007 - 2008}{Software Engineer}{Intelligent Information Systems}{Durham, NC}{}{
    \begin{itemize}
      % \item Helped design and implement a web application that allows
      %   some North Carolina counties to handle and process appeals
      %   related to taxes on property made by residents of these
      %   counties using C\#, ASP.NET, MS SQL and Javascript 
      \item Designed and implemented a website so clients could
        access and browse property inspection information 
      \item Implemented and improved a procedure to port and test
        programs written in VB6 to VB.NET. Converted over 100 programs
        written in VB6 code to VB.NET using a combination of the MS
        conversion tool and Python. 
    \end{itemize}
  }

\section{Education}
\cventry {Fall 2006 - Winter 2008}{MS Applied Mathematics, North
  Carolina State University} {Raleigh, NC}{}{}{}
\cventry{Fall 2001 - Winter 2006}{BS Computer Science, North Carolina State University}{Raleigh, NC}{}{}{}
\cventry{Fall 2001 - Winter 2006}{BS Applied Mathematics, North Carolina State University}{Raleigh, NC}{}{}{}

\section{Extra}
\cvlistitem{Presenter,PyData: ``Wikipedia Indexing and Analysis'' - see \url{https://vimeo.com/53091620}}
\cvlistitem{Contributor to open source projects - see
    \url{https://github.com/dfdeshom}}

\end{document}
